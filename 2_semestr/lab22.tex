\documentclass[10pt, a5paper]{book}
\usepackage[utf8]{inputenc}
\usepackage[T2A]{fontenc}
\usepackage[english,russian]{babel}
\usepackage{amsmath}
\usepackage{amsthm}
\usepackage{fancyhdr}
\usepackage{setspace}
\usepackage[left=1cm,right=1.5cm,top=2mm,bottom=5mm,bindingoffset=0.5cm]{geometry}
\newcommand{\RomanNumeralCaps}[1]
{\MakeUppercase{\romannumeral #1}} 
\newtheoremstyle{remboldstyle}
{}{}{\itshape}{\parindent}{}{}{.5em}{}
\theoremstyle{remboldstyle} 
\newtheorem{theorem}{\hspace{1mm} \normalsizeТ {\scriptsize Е О Р Е М А} }
\parindent=1.2cm
\pagestyle{empty}
\begin{document}
\normalsize
\pagestyle{fancy}
\begin{spacing}{2}
{\noindent 258\;\;\;\;\;\;\;\scriptsize СИСТЕМЫ ОБЫКНОВЕННЫХ ДИФФЕРЕНЦИАЛЬНЫХ УРАВНЕНИЙ [ГЛ. VI}
\end{spacing}
\renewcommand{\thetheorem}{\arabic{theorem}.}
\renewcommand{\headrulewidth}{0pt} 
\noindentв интервале {\itshape (a, b)}, то, деля на {\itshape D(x)},\;\;\;получаем линейную систему уравнений в нормальной форме.

{\itshape Пример 3.} Найти линейную однородную систему второго порядка, допускающую следующую систему решений:
$$
\left.
\begin{array}{llr}
	y_1^{(1)}=e^{x}\cdot\cos x, & y_2^{(1)}=e^{x}\cdot\sin x;\\ 
	y_1^{(2)}=-\sin x, & y_2^{(2)}=\cos x.
\end{array}
\right.
$$
Искомые уравнения будут:
$$
\left|
\begin{array}{llr}
	\frac{dy_1}{dx} & e^{x}(\cos x - \sin x) & - \cos x\\
	y_1 & e^{x}\cos x & - \sin x\\
	y_2 & e^{x}\sin x & \cos x
\end{array}
\right|=0{,}
$$
$$
\left|
\begin{array}{llr}
        \frac{dy_2}{dx} & e^{x}(\sin x+\cos x) & -\sin x\\
        y_1 & e^{x}\cos x & -\sin x\\
        y_2 & e^{x}\sin x & \cos x
\end{array}
\right|=0{,}
$$
или, развертывая определители по первому столбцу и деля оба уравненения на $D(x)=e^{x}$, получаем искомую систему:
$$
\frac{dy_1}{dx}-\cos^{2} x\cdot y_1+(1-\sin x \cdot\cos x)y_2=0{,}
$$
$$
\frac{dy_2}{dx}-(1+\sin x\cdot\cos x)y_1-\sin^2 x\cdot y_2=0{.}
$$

3. Н\,е\,о\,д\,н\,о\,р\,о\,д\,н\,ы\,е с\,и\,с\,т\,е\,м\,ы л\,и\,н\,е\,й\,н\,ы\,х у\,р\,а\,в\,н\,е\,н\,и\,й. Рассмотрим неоднородную систему:

{\setstretch{1.3}
$$
\left.
\begin{array}{c}
	\frac{dy_1}{dx}+a_{11}y_1+a_{12}y_2+\hdots +a_{1n}y_n=V_{1},\\
	\frac{dy_2}{dx}+a_{21}y_1+a_{22}y_2+\hdots +a_{2n}y_n=V_{2},\\
	\hdotsfor{1}\\
	\hdotsfor{1}\\
	\frac{dy_n}{dx}+a_{n1}y_1+a_{n2}y_2+\hdots +a_{nn}y_n=V_{n}.
\end{array}
\right\} \eqno (8)
$$
}

	Т\,{\scriptsize Е\,О\,Р\,Е\,М\,А}\;\;\; 1.\;\; {\itshapeЕсли известно частное решение неоднородной системы: $Y_1(x),\;\;Y_2(x),\;\;\dots,\;\;Y_n(x)$, то нахождение общего решения этой системы приводится\; к\; решению\; соответствующей однородной системы} (9).

В самом деле, введем новые искомые функции $z_i$ соотношениями
$$y_1=Y_1+z_1,\;\; y_2=Y_2+z_2,\;\; \dots ,\;\; y_n=Y_n+z_n.$$
Внося\;\; эти\;\; выражения\;\; в\;\; уравнения\;\; (8)\;\; и\;\; принимая\;\; во\;\; внимание\; тождества
$$
{\frac{dY_1}{dx}+a_{i1}Y_1+a_{i2}Y_2 +}\, \dots\, {+ a_{in}Y_n=V_i(x),\; (i=1, 2,\, \dots , n),}
$$
\begin{spacing}{2}
\noindent{\scriptsize \textsection 2] СИСТЕМЫ ОБЫКНОВЕННЫХ ДИФФЕРЕНЦИАЛЬНЫХ УРАВНЕНИЙ}\;\;\;\;\;\;\;\;\;\;\;\;\;\;259
\end{spacing}
\noindent мы получим для новых функция $z_i$ систему:
$$
\setlength\abovedisplayskip{0pt}
{\frac{dz_1}{dx}+a_{i1}z_1+a_{i2}z_2+} \dots {+a_{in}z_n=0,{\;} {\;} (i = 1,\, 2,\, 3,\, \dots,\, n).}\eqno (9\prime)
$$
Теорема доказана.

{\hspace{1mm} C л е д с т в и е.} Общее решение системы (8) имеет вид:
	$$
	\setlength\abovedisplayskip{0pt}
	\left.
\begin{array}{c}
	y_1=C_1y_1^{(1)}+C_2y_1^{(2)}{+}\dots{+}C_ny_1^{(n)}+Y_1,\\
	y_2=C_1y_2^{(1)}+C_2y_2^{(2)}{+}\dots{+}C_ny_2^{(n)}+Y_2,\\
	\hdotsfor{1}\\
	\hdotsfor{1}\\
	y_n=C_1y_n^{(1)}+C_2y_n^{(2)}{+}\dots{+}C_ny_n^{(n)}+Y_n,
\end{array} 
\right.
	$$
	где $Y_1,\; Y_2,\; \dots,\; Y_n$\, --- какое-нибудь\;\; частное\;\; решение\;\; неоднородной системы (8),\; а
	$$
	\setlength\abovedisplayskip{0pt}
	y_1^{(1)}, y_2^{(1)},\, \dots,\, y_n^{(1)};\; y_1^{(2)},\, y_2^{(2)},\, \dots,\, y_n^{(2)};\; \dots; 
	$$
	$$
	y_1{(n)}, y_2{(n)}, \dots, y_n{(n)} \eqno(12)
	$$
	суть\; {\itshape n}\; независимых\; частных\; решений\; соответствующей\; однородной\; системы (9); $C_1, C_2, \,\dots, C_n$ --- произвольные постоянные. Доказательство аналогично соответствующему доказательству теоремы для линейного уравнения {\itshape n}-го порядка (см. гл. \RomanNumeralCaps{5}).

	Т\,{\scriptsize Е\,О\,Р\,Е\,М\,А}\;\;\; 2.\;\;{\itshapeЕсли известна система n независимых решений соответствующей однородной линейной системы, то решение неоднородной системы сводится к квадратурам.}


Если\,нам\,известны\,решения (12) системы (9), то ее общеее решение имеет вид:
$$
\setlength\abovedisplayskip{0pt}
\left.
\begin{array}{c}
	y_1=C_1y_1^{(1)}+C_2y_1^{(2)}+{\dots}+C_ny_1^{(n)},\\
	y_2=C_1y_2^{(1)}+C_2y_2^{(2)}+{\dots}+C_ny_2^{(n)},\\
	\hdotsfor{1}\\
	\hdotsfor{1}\\
	y_n=C_1y_n^{(1)}+C_2y_n^{(2)}+{\dots}+C_ny_n^{(n)},
\end{array} 
\right\} \eqno (13)
$$
где $C_1,\, C_2,\, \dots,\, C_n$ --- постоянные.\;\; Формулы (13)\;\; с постоянными $C_i$,\,\;\; очевидно, не дают решения неоднородной системы (8). Применим,\; как\: и\: в случае\: одного\: линейного\:\: уравнения, м\,е\,т\,о\,д\;\;\;в\,а\,р\,и\,а\,ц\,и\,и\;\;\; п\,о- с\,т\,о\,я\,н\,н\,ы\,х. Будем\; рассматривать\; $C_i$\; как\: неизвестные функции от $x$, причем подберем их таким образом, чтобы выражения\; (13)\;\; являлись\; решениями неоднородной системы [систему уравнений (13) можно рассматривать\; как\; систему,\;\; вводящую $n$ новых\;\; искомых\;\; функций от $x,\\ C_1,\, C_2,\, \dots,\, C_n$; в силу линейности\; преобразования\; новые\; уравнения\;\; для $C_i$ тоже будут линейными].

{\setstretch{1.3}
Дифференцируем равеенства (13) по {\itshape x}:\newline 

$
\frac{dy_i}{dx}=C_1\frac{dy_1^{(1)}}{dx}+C_2\frac{dy_i^{(2)}}{dx}+{\dots}+C_n\frac{dy_i^{(n)}}{dx}+
$
$$
\;+y_i^{(1)}\frac{dC_1}{dx}+y_i^{(2)}+{\dots}+y_i^{(n)}\frac{dC_n}{dx} (i=1, 2, {\dots}, n). \eqno(17)
$$
}
\end{document}

