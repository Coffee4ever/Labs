\documentclass{book}
\usepackage[utf8]{inputenc}
\usepackage[T2A]{fontenc}
\usepackage{amsmath}
\usepackage{amsthm}
\usepackage{fancyhdr}
\usepackage{setspace}
\newcommand{\RomanNumeralCaps}[1]
{\MakeUppercase{\romannumeral #1}} 
\newtheoremstyle{remboldstyle}
{}{}{}{\parindent}{}{}{.5em}{{\thmname{#1 }}{\thmnumber{#2}}{\thmnote{ (#3)}}}
\theoremstyle{remboldstyle} 
\newtheorem{theorem}{\normalsizeТ{\scriptsize ЕОРЕМА}}
\textwidth=118mm
\begin{document}
\setcounter{page}{258}
\pagestyle{fancy}
\fancyhead{} 
\fancyhead[LE,RO]{\thepage} 
\fancyhead[CE]{\tiny СИСТЕМЫ ОБЫКНОВЕННЫХ ДИФФЕРЕНЦИАЛЬНЫХ УРАВНЕНИЙ [ГЛ. VI}
\renewcommand{\thetheorem}{\arabic{theorem}.}
\renewcommand{\headrulewidth}{0pt} 
\fancyhead[CO]{\scriptsize \textsection 2] СИСТЕМЫ ОБЫКНОВЕННЫХ ДИФФЕРЕНЦИАЛЬНЫХ УРАВНЕНИЙ} 
\fancyfoot{} 
\noindentв интервале {\itshape (a, b)}, то, деля на {\itshape D(x)}, получаем линейную систему уравнений в нормальной форме.

{\itshape Пример 3.} Найти линейную однородную систему второго порядка, допускающую следующую систему решений:
$$
\left.
\begin{array}{llr}
	y_1^{(1)}=e^{x}\cdot\cos x, & y_2^{(1)}=e^{x}\cdot\sin x;\\ 
	y_1^{(2)}=-\sin x, & y_2^{(2)}=\cos x.
\end{array}
\right.
$$
Искомые уравнения будут:
$$
\left|
\begin{array}{llr}
	\frac{dy_1}{dx} & e^{x}(\cos x - \sin x) & - \cos x\\
	y_1 & e^{x}\cos x & - \sin x\\
	y_2 & e^{x}\sin x & \cos x
\end{array}
\right|=0{,}
$$
$$
\left|
\begin{array}{llr}
        \frac{dy_2}{dx} & e^{x}(\sin x+\cos x) & -\sin x\\
        y_1 & e^{x}\cos x & -\sin x\\
        y_2 & e^{x}\sin x & \cos x
\end{array}
\right|=0{,}
$$
или, развертывая определители по первому столбцу и деля оба уравненения на $D(x)=e^{x}$, получаем искомую систему:
$$
\frac{dy_1}{dx}-\cos^{2} x\cdot y_1+(1-\sin x \cdot\cos x)y_2=0{,}
$$
$$
\frac{dy_2}{dx}-(1+\sin x\cdot\cos x)y_1-\sin^2 x\cdot y_2=0{.}
$$
3.{\hspace{1mm} Н е о д н о р о д н ы е} {\hspace{1mm} с и с т е м ы} {\hspace{1mm} л и н е й н ы х} {\hspace{1mm} у р а в н е н и й}. Рассмотрим неоднородную систему:

{\setstretch{1.3}
$$
\left.
\begin{array}{c}
	\frac{dy_1}{dx}+a_{11}y_1+a_{12}y_2+\cdots +a_{1n}y_n=V_{1},\\
	\frac{dy_2}{dx}+a_{21}y_1+a_{22}y_2+\cdots +a_{2n}y_n=V_{2},\\
	\hdotsfor{1}\\
	\hdotsfor{1}\\
	\frac{dy_n}{dx}+a_{n1}y_1+a_{n2}y_2+\cdots +a_{nn}y_n=V_{n}.
\end{array}
\right\} \eqno (8)
$$
}
\begin{theorem}
	Если известно частное решение неоднородной системы: $Y_1(x),{ }Y_2(x), \cdots, Y_n(x)$, то нахождение общего решения этой системы приводится к решению соответствующей однородной системы (9).
\end{theorem}

В самом деле, введём новые искомые функции $z_i$ соотношениями
$$y_1=Y_1+z_1, y_2=Y_2+z_2, \cdots , y_n=Y_n+z_n.$$
Внося эти выражения в уравнения (8) и принимая во внимание тождества
$$
\frac{dY_1}{dx}+a_{i1}Y_1+a_{i2}Y_2+ \cdots +a_{in}Y_n=V_i(x), (i=1, 2, \cdots , n),
$$
мы получим для новых функция $z_i$ систему:
$$
\frac{dz_1}{dx}+a_{i1}z_1+a_{i2}z_2+\cdots +a_{in}z_n=0, (i = 1, 2, 3, \cdots, n).\eqno (9\prime)
$$
Теорема доказана.

{\hspace{1mm} C л е д с т в и е.} Общее решение системы (8) имеет вид:
	$$
	\left.
\begin{array}{c}
	y_1=C_1y_1^{(1)}+C_2y_1^{(2)}+\cdots+C_ny_1^{(n)}+Y_1,\\
	y_2=C_1y_2^{(1)}+C_2y_2^{(2)}+\cdots+C_ny_2^{(n)}+Y_2,\\
	\hdotsfor{1}\\
	\hdotsfor{1}\\
	y_n=C_1y_n^{(1)}+C_2y_n^{(2)}+\cdots+C_ny_n^{(n)}+Y_n,
\end{array} 
\right.
	$$
	где $Y_1, Y_2, \cdots, Y_n$ - какое-нибудь частное решение неоднородной системы (8), а
	$$
	y_1^{(1)}, y_2^{(1)}, \cdots, y_n^{(1)}; y_1^{(2)}, y_2^{(2)}, \cdots, y_n^{(2)}; \cdots; \newline
	y_1{(n)}, y_2{(n)}, \cdots, y_n{(n)} \eqno(12)
	$$
	суть {\itshape n} независимых частных решений соответствующей однородной системы (9); $C_1, C_2, \cdots, C_n$ - произвольные постоянные. Доказательство аналогично соответствующему доказательству теоремы для линейного уравнения {\itshape n}-го порядка (см. гл. \RomanNumeralCaps{5}).
\begin{theorem}
	Если известна система n независимых решений соответствующей однородной линейной системы, то решение неоднородной системы сводится к квадратурам.
\end{theorem}

Если нам известны решения (12) системы (9), то ее общеее решение имеет вид:
$$
\left.
\begin{array}{c}
	y_1=C_1y_1^{(1)}+C_2y_1^{(2)}+\cdots+C_ny_1^{(n)},\\
	y_2=C_1y_2^{(1)}+C_2y_2^{(2)}+\cdots+C_ny_2^{(n)},\\
	\hdotsfor{1}\\
	\hdotsfor{1}\\
	y_n=C_1y_n^{(1)}+C_2y_n^{(2)}+\cdots+C_ny_n^{(n)},
\end{array} 
\right\} \eqno (13)
$$
где $С_1, C_2, \cdots, C_n$ - постоянные. Формулы (13) с постоянными $C_i$, очевидно, не дают решения неоднородной системы (8). Применим, как и в случае одного линейного уравнения, {\hspace{1mm} м е т о д} {\hspace{1mm} в а р и а ц и и} {\hspace{1mm} п о с т о я н н ы х}. Будем рассматривать $C_i$ как неизвестные функции от x, причем подберем их таким образом, чтобы выражения (13) являлись решениями неоднородной системы [систему уравнений (13) можно рассматривать как систему, вводящую n новых искомых функций от $x, C_1, C_2, \cdots, C_n$; в силу линейности преобразования новые уравнения для $C_i$ тоже будут линейными].

Дифференцируем равеенства (13) по {\itshape x}:
$$
\frac{dy_i}{dx}=C_1\frac{dy_1^{(1)}}{dx}+C_2\frac{dy_i^{(2)}}{dx}+\cdots+C_n\frac{dy_i^{(n)}}{dx}+\newline
+y_i^{(1)}\frac{dC_1}{dx}+y_i^{(2)}+\cdots+y_i^{(n)}\frac{dC_n}{dx} (i=1, 2, \cdots, n). \eqno(17)
$$

\end{document}

